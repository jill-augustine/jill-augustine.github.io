\documentclass[]{book}
\usepackage{lmodern}
\usepackage{amssymb,amsmath}
\usepackage{ifxetex,ifluatex}
\usepackage{fixltx2e} % provides \textsubscript
\ifnum 0\ifxetex 1\fi\ifluatex 1\fi=0 % if pdftex
  \usepackage[T1]{fontenc}
  \usepackage[utf8]{inputenc}
\else % if luatex or xelatex
  \ifxetex
    \usepackage{mathspec}
  \else
    \usepackage{fontspec}
  \fi
  \defaultfontfeatures{Ligatures=TeX,Scale=MatchLowercase}
\fi
% use upquote if available, for straight quotes in verbatim environments
\IfFileExists{upquote.sty}{\usepackage{upquote}}{}
% use microtype if available
\IfFileExists{microtype.sty}{%
\usepackage{microtype}
\UseMicrotypeSet[protrusion]{basicmath} % disable protrusion for tt fonts
}{}
\usepackage[margin=1in]{geometry}
\usepackage{hyperref}
\hypersetup{unicode=true,
            pdfborder={0 0 0},
            breaklinks=true}
\urlstyle{same}  % don't use monospace font for urls
\usepackage{natbib}
\bibliographystyle{plainnat}
\usepackage{longtable,booktabs}
\usepackage{graphicx,grffile}
\makeatletter
\def\maxwidth{\ifdim\Gin@nat@width>\linewidth\linewidth\else\Gin@nat@width\fi}
\def\maxheight{\ifdim\Gin@nat@height>\textheight\textheight\else\Gin@nat@height\fi}
\makeatother
% Scale images if necessary, so that they will not overflow the page
% margins by default, and it is still possible to overwrite the defaults
% using explicit options in \includegraphics[width, height, ...]{}
\setkeys{Gin}{width=\maxwidth,height=\maxheight,keepaspectratio}
\IfFileExists{parskip.sty}{%
\usepackage{parskip}
}{% else
\setlength{\parindent}{0pt}
\setlength{\parskip}{6pt plus 2pt minus 1pt}
}
\setlength{\emergencystretch}{3em}  % prevent overfull lines
\providecommand{\tightlist}{%
  \setlength{\itemsep}{0pt}\setlength{\parskip}{0pt}}
\setcounter{secnumdepth}{5}
% Redefines (sub)paragraphs to behave more like sections
\ifx\paragraph\undefined\else
\let\oldparagraph\paragraph
\renewcommand{\paragraph}[1]{\oldparagraph{#1}\mbox{}}
\fi
\ifx\subparagraph\undefined\else
\let\oldsubparagraph\subparagraph
\renewcommand{\subparagraph}[1]{\oldsubparagraph{#1}\mbox{}}
\fi

%%% Use protect on footnotes to avoid problems with footnotes in titles
\let\rmarkdownfootnote\footnote%
\def\footnote{\protect\rmarkdownfootnote}

%%% Change title format to be more compact
\usepackage{titling}

% Create subtitle command for use in maketitle
\providecommand{\subtitle}[1]{
  \posttitle{
    \begin{center}\large#1\end{center}
    }
}

\setlength{\droptitle}{-2em}

  \title{}
    \pretitle{\vspace{\droptitle}}
  \posttitle{}
    \author{}
    \preauthor{}\postauthor{}
    \date{}
    \predate{}\postdate{}
  
\usepackage{booktabs}
\usepackage{amsthm}
\makeatletter
\def\thm@space@setup{%
  \thm@preskip=8pt plus 2pt minus 4pt
  \thm@postskip=\thm@preskip
}
\makeatother

\begin{document}

{
\setcounter{tocdepth}{1}
\tableofcontents
}
\hypertarget{home}{%
\chapter*{Home}\label{home}}
\addcontentsline{toc}{chapter}{Home}

Welcome to my blog! This is where I share excerpts of my portfolio as well as professional resources I am currently using. I will also share some blog posts about experiences from my working life.

\hypertarget{latest-projects}{%
\section{Latest Projects}\label{latest-projects}}

\begin{itemize}
\tightlist
\item
  {[}Building a Data Exploration Web App with Shiny{]}{[}data-exploration-using-shiny{]}
\end{itemize}

\hypertarget{about-me}{%
\chapter*{About Me}\label{about-me}}
\addcontentsline{toc}{chapter}{About Me}

Jillian Augustine, PhD.

I am a data scientist and question answerer. I received my PhD in Molecular Biology from the University of Vienna (Austria) and my BSc from the University of Leeds (UK) during which I also studied at McGill University (Montreal, Canada).

My main interest (and the reason I became a data scientist) is using data to answer questions regardless of the industry. I have experience working in both telecommunications and the Paper \& Packaging industries. In particular I am passionate about data understanding and communication both to stakeholders and within data teams. My professional approach to data science is to use the tool that gets the job done given any constraints from team members and stakeholders. Here is a list of my current go-to tools.

Data Manipulation:

\begin{itemize}
\tightlist
\item
  python
\item
  Apache Spark
\item
  R
\item
  any combination of the above
\item
  Microsoft Excel (for team members accustomed to pivot tables)
\end{itemize}

(Real-Time) Data Collection/Extraction:

\begin{itemize}
\tightlist
\item
  SQL (Hive)
\item
  Apache Kafka
\end{itemize}

Data Storage:

\begin{itemize}
\tightlist
\item
  parquet files whenever possible
\item
  Hadoop Distributed File System (HDFS)
\item
  Linux computer clusters
\end{itemize}

Data Visualisation:

\begin{itemize}
\tightlist
\item
  ggplot2
\end{itemize}

Machine Learning:

\begin{itemize}
\tightlist
\item
  caret
\item
  (scikit-learn)
\end{itemize}

Documentation/Project Work:

\begin{itemize}
\tightlist
\item
  R Markdown
\item
  Confluence
\item
  Jira
\item
  Git
\end{itemize}

Another interest of mine is increasing the inclusivity of working groups through open exchanges and active diversificartion. I try to make my presentations are accessible as possible and welcome feedback as to how I can improve this further. I welcome opportunities to speak about my work at conferences and meetups. Please get in touch through \href{https://twitter.com/jill_augustine}{Twitter} or \href{https://www.linkedin.com/in/jillianaugustine/}{LinkedIn} if you would like to know more.

\hypertarget{data-exploration-with-shiny}{%
\chapter{Data Exploration with Shiny}\label{data-exploration-with-shiny}}

Why?

How?

My ideas
- email notification

Requirements from users?

Requirements from the business
- logging of users requests and viewing
- automatic deletion of files
- limited access to data

Prototypes?

\hypertarget{useful-links}{%
\chapter{Useful Links}\label{useful-links}}

I am often asked about which resources I use to improve and maintain my data science skills. Here is a list of websites I often use and would recommend.

\hypertarget{applications}{%
\chapter{Applications}\label{applications}}

Some \emph{significant} applications are demonstrated in this chapter.

\hypertarget{example-one}{%
\section{Example one}\label{example-one}}

\hypertarget{example-two}{%
\section{Example two}\label{example-two}}

\hypertarget{final-words}{%
\chapter{Final Words}\label{final-words}}

We have finished a nice book.


\end{document}
